\documentclass[]{article}
\usepackage{lmodern}
\usepackage{amssymb,amsmath}
\usepackage{ifxetex,ifluatex}
\usepackage{fixltx2e} % provides \textsubscript
\ifnum 0\ifxetex 1\fi\ifluatex 1\fi=0 % if pdftex
  \usepackage[T1]{fontenc}
  \usepackage[utf8]{inputenc}
\else % if luatex or xelatex
  \ifxetex
    \usepackage{mathspec}
  \else
    \usepackage{fontspec}
  \fi
  \defaultfontfeatures{Ligatures=TeX,Scale=MatchLowercase}
\fi
% use upquote if available, for straight quotes in verbatim environments
\IfFileExists{upquote.sty}{\usepackage{upquote}}{}
% use microtype if available
\IfFileExists{microtype.sty}{%
\usepackage{microtype}
\UseMicrotypeSet[protrusion]{basicmath} % disable protrusion for tt fonts
}{}
\usepackage[margin=1in]{geometry}
\usepackage{hyperref}
\hypersetup{unicode=true,
            pdftitle={GGplot-Iris.R},
            pdfauthor={34320},
            pdfborder={0 0 0},
            breaklinks=true}
\urlstyle{same}  % don't use monospace font for urls
\usepackage{color}
\usepackage{fancyvrb}
\newcommand{\VerbBar}{|}
\newcommand{\VERB}{\Verb[commandchars=\\\{\}]}
\DefineVerbatimEnvironment{Highlighting}{Verbatim}{commandchars=\\\{\}}
% Add ',fontsize=\small' for more characters per line
\usepackage{framed}
\definecolor{shadecolor}{RGB}{248,248,248}
\newenvironment{Shaded}{\begin{snugshade}}{\end{snugshade}}
\newcommand{\KeywordTok}[1]{\textcolor[rgb]{0.13,0.29,0.53}{\textbf{#1}}}
\newcommand{\DataTypeTok}[1]{\textcolor[rgb]{0.13,0.29,0.53}{#1}}
\newcommand{\DecValTok}[1]{\textcolor[rgb]{0.00,0.00,0.81}{#1}}
\newcommand{\BaseNTok}[1]{\textcolor[rgb]{0.00,0.00,0.81}{#1}}
\newcommand{\FloatTok}[1]{\textcolor[rgb]{0.00,0.00,0.81}{#1}}
\newcommand{\ConstantTok}[1]{\textcolor[rgb]{0.00,0.00,0.00}{#1}}
\newcommand{\CharTok}[1]{\textcolor[rgb]{0.31,0.60,0.02}{#1}}
\newcommand{\SpecialCharTok}[1]{\textcolor[rgb]{0.00,0.00,0.00}{#1}}
\newcommand{\StringTok}[1]{\textcolor[rgb]{0.31,0.60,0.02}{#1}}
\newcommand{\VerbatimStringTok}[1]{\textcolor[rgb]{0.31,0.60,0.02}{#1}}
\newcommand{\SpecialStringTok}[1]{\textcolor[rgb]{0.31,0.60,0.02}{#1}}
\newcommand{\ImportTok}[1]{#1}
\newcommand{\CommentTok}[1]{\textcolor[rgb]{0.56,0.35,0.01}{\textit{#1}}}
\newcommand{\DocumentationTok}[1]{\textcolor[rgb]{0.56,0.35,0.01}{\textbf{\textit{#1}}}}
\newcommand{\AnnotationTok}[1]{\textcolor[rgb]{0.56,0.35,0.01}{\textbf{\textit{#1}}}}
\newcommand{\CommentVarTok}[1]{\textcolor[rgb]{0.56,0.35,0.01}{\textbf{\textit{#1}}}}
\newcommand{\OtherTok}[1]{\textcolor[rgb]{0.56,0.35,0.01}{#1}}
\newcommand{\FunctionTok}[1]{\textcolor[rgb]{0.00,0.00,0.00}{#1}}
\newcommand{\VariableTok}[1]{\textcolor[rgb]{0.00,0.00,0.00}{#1}}
\newcommand{\ControlFlowTok}[1]{\textcolor[rgb]{0.13,0.29,0.53}{\textbf{#1}}}
\newcommand{\OperatorTok}[1]{\textcolor[rgb]{0.81,0.36,0.00}{\textbf{#1}}}
\newcommand{\BuiltInTok}[1]{#1}
\newcommand{\ExtensionTok}[1]{#1}
\newcommand{\PreprocessorTok}[1]{\textcolor[rgb]{0.56,0.35,0.01}{\textit{#1}}}
\newcommand{\AttributeTok}[1]{\textcolor[rgb]{0.77,0.63,0.00}{#1}}
\newcommand{\RegionMarkerTok}[1]{#1}
\newcommand{\InformationTok}[1]{\textcolor[rgb]{0.56,0.35,0.01}{\textbf{\textit{#1}}}}
\newcommand{\WarningTok}[1]{\textcolor[rgb]{0.56,0.35,0.01}{\textbf{\textit{#1}}}}
\newcommand{\AlertTok}[1]{\textcolor[rgb]{0.94,0.16,0.16}{#1}}
\newcommand{\ErrorTok}[1]{\textcolor[rgb]{0.64,0.00,0.00}{\textbf{#1}}}
\newcommand{\NormalTok}[1]{#1}
\usepackage{graphicx,grffile}
\makeatletter
\def\maxwidth{\ifdim\Gin@nat@width>\linewidth\linewidth\else\Gin@nat@width\fi}
\def\maxheight{\ifdim\Gin@nat@height>\textheight\textheight\else\Gin@nat@height\fi}
\makeatother
% Scale images if necessary, so that they will not overflow the page
% margins by default, and it is still possible to overwrite the defaults
% using explicit options in \includegraphics[width, height, ...]{}
\setkeys{Gin}{width=\maxwidth,height=\maxheight,keepaspectratio}
\IfFileExists{parskip.sty}{%
\usepackage{parskip}
}{% else
\setlength{\parindent}{0pt}
\setlength{\parskip}{6pt plus 2pt minus 1pt}
}
\setlength{\emergencystretch}{3em}  % prevent overfull lines
\providecommand{\tightlist}{%
  \setlength{\itemsep}{0pt}\setlength{\parskip}{0pt}}
\setcounter{secnumdepth}{0}
% Redefines (sub)paragraphs to behave more like sections
\ifx\paragraph\undefined\else
\let\oldparagraph\paragraph
\renewcommand{\paragraph}[1]{\oldparagraph{#1}\mbox{}}
\fi
\ifx\subparagraph\undefined\else
\let\oldsubparagraph\subparagraph
\renewcommand{\subparagraph}[1]{\oldsubparagraph{#1}\mbox{}}
\fi

%%% Use protect on footnotes to avoid problems with footnotes in titles
\let\rmarkdownfootnote\footnote%
\def\footnote{\protect\rmarkdownfootnote}

%%% Change title format to be more compact
\usepackage{titling}

% Create subtitle command for use in maketitle
\newcommand{\subtitle}[1]{
  \posttitle{
    \begin{center}\large#1\end{center}
    }
}

\setlength{\droptitle}{-2em}

  \title{GGplot-Iris.R}
    \pretitle{\vspace{\droptitle}\centering\huge}
  \posttitle{\par}
    \author{34320}
    \preauthor{\centering\large\emph}
  \postauthor{\par}
      \predate{\centering\large\emph}
  \postdate{\par}
    \date{Wed Jan 23 17:36:50 2019}


\begin{document}
\maketitle

\begin{Shaded}
\begin{Highlighting}[]
\KeywordTok{library}\NormalTok{(ggplot2)}
\end{Highlighting}
\end{Shaded}

\begin{verbatim}
## Warning: package 'ggplot2' was built under R version 3.5.2
\end{verbatim}

\begin{Shaded}
\begin{Highlighting}[]
\NormalTok{iris <-}\StringTok{ }\KeywordTok{read.csv}\NormalTok{(}\KeywordTok{url}\NormalTok{(}\StringTok{"http://archive.ics.uci.edu/ml/machine-learning-databases/iris/iris.data"}\NormalTok{), }\DataTypeTok{header =} \OtherTok{FALSE}\NormalTok{)}
\KeywordTok{names}\NormalTok{(iris) <-}\StringTok{ }\KeywordTok{c}\NormalTok{( }\StringTok{"Petal.Length"}\NormalTok{, }\StringTok{"Petal.Width"}\NormalTok{,}\StringTok{"Sepal.Length"}\NormalTok{, }\StringTok{"Sepal.Width"}\NormalTok{, }\StringTok{"Species"}\NormalTok{)}
\CommentTok{#iris}
\NormalTok{IrisPlot <-}\KeywordTok{ggplot}\NormalTok{(iris, }\KeywordTok{aes}\NormalTok{(}\DataTypeTok{x=}\NormalTok{Petal.Length, }\DataTypeTok{y=}\NormalTok{Petal.Width))}
\KeywordTok{plot}\NormalTok{(IrisPlot)}
\end{Highlighting}
\end{Shaded}

\includegraphics{GGplot-Iris_files/figure-latex/unnamed-chunk-1-1.pdf}

\begin{Shaded}
\begin{Highlighting}[]
\CommentTok{#PetalPlot<-ggplot(iris, aes(x=Petal.Length, y=Petal.Width)) + geom_point()}
\CommentTok{#plot(PetalPlot)}
\NormalTok{IrisPlot <-}\StringTok{ }\NormalTok{IrisPlot }\OperatorTok{+}\StringTok{  }\KeywordTok{geom_point}\NormalTok{()}
\KeywordTok{plot}\NormalTok{(IrisPlot)}
\end{Highlighting}
\end{Shaded}

\includegraphics{GGplot-Iris_files/figure-latex/unnamed-chunk-1-2.pdf}

\begin{Shaded}
\begin{Highlighting}[]
\NormalTok{################}
\CommentTok{#linear plotting}
\NormalTok{IrisPlot <-}\StringTok{ }\NormalTok{IrisPlot }\OperatorTok{+}\StringTok{  }\KeywordTok{geom_point}\NormalTok{() }\OperatorTok{+}\StringTok{  }\KeywordTok{geom_smooth}\NormalTok{(}\DataTypeTok{method=}\StringTok{"lm"}\NormalTok{)  }\CommentTok{# set se=FALSE to turnoff confidence bands}
\KeywordTok{plot}\NormalTok{(IrisPlot)}
\end{Highlighting}
\end{Shaded}

\includegraphics{GGplot-Iris_files/figure-latex/unnamed-chunk-1-3.pdf}

\begin{Shaded}
\begin{Highlighting}[]
\CommentTok{# Delete the points outside the limits}

\NormalTok{IrisPlot <-}\StringTok{ }\NormalTok{IrisPlot }\OperatorTok{+}\StringTok{ }\KeywordTok{xlim}\NormalTok{(}\KeywordTok{c}\NormalTok{(}\FloatTok{2.0}\NormalTok{, }\FloatTok{7.0}\NormalTok{)) }\OperatorTok{+}\StringTok{ }\KeywordTok{ylim}\NormalTok{(}\KeywordTok{c}\NormalTok{(}\FloatTok{2.5}\NormalTok{, }\FloatTok{3.5}\NormalTok{))   }\CommentTok{#  deletes points}
\KeywordTok{plot}\NormalTok{(IrisPlot)}
\end{Highlighting}
\end{Shaded}

\begin{verbatim}
## Warning: Removed 38 rows containing non-finite values (stat_smooth).
\end{verbatim}

\begin{verbatim}
## Warning: Removed 38 rows containing missing values (geom_point).

## Warning: Removed 38 rows containing missing values (geom_point).
\end{verbatim}

\includegraphics{GGplot-Iris_files/figure-latex/unnamed-chunk-1-4.pdf}

\begin{Shaded}
\begin{Highlighting}[]
\NormalTok{IrisPlot <-}\StringTok{ }\NormalTok{IrisPlot }\OperatorTok{+}\StringTok{ }\KeywordTok{xlim}\NormalTok{(}\KeywordTok{c}\NormalTok{(}\FloatTok{0.0}\NormalTok{, }\FloatTok{8.0}\NormalTok{)) }\OperatorTok{+}\StringTok{ }\KeywordTok{ylim}\NormalTok{(}\KeywordTok{c}\NormalTok{(}\FloatTok{0.0}\NormalTok{, }\FloatTok{8.0}\NormalTok{))   }\CommentTok{#  deletes points}
\end{Highlighting}
\end{Shaded}

\begin{verbatim}
## Scale for 'x' is already present. Adding another scale for 'x', which
## will replace the existing scale.
\end{verbatim}

\begin{verbatim}
## Scale for 'y' is already present. Adding another scale for 'y', which
## will replace the existing scale.
\end{verbatim}

\begin{Shaded}
\begin{Highlighting}[]
\KeywordTok{plot}\NormalTok{(IrisPlot)}
\end{Highlighting}
\end{Shaded}

\includegraphics{GGplot-Iris_files/figure-latex/unnamed-chunk-1-5.pdf}

\begin{Shaded}
\begin{Highlighting}[]
\NormalTok{################}
\CommentTok{#Zooming In}
\CommentTok{#Reassign the Iris data since it's already been eliminated by some limits'}

\NormalTok{IrisPlot <-}\KeywordTok{ggplot}\NormalTok{(iris, }\KeywordTok{aes}\NormalTok{(}\DataTypeTok{x=}\NormalTok{Petal.Length, }\DataTypeTok{y=}\NormalTok{Petal.Width))   }\OperatorTok{+}\StringTok{  }\KeywordTok{geom_point}\NormalTok{() }\OperatorTok{+}\StringTok{  }\KeywordTok{geom_smooth}\NormalTok{(}\DataTypeTok{method=}\StringTok{"lm"}\NormalTok{)}

\KeywordTok{plot}\NormalTok{(IrisPlot)}
\end{Highlighting}
\end{Shaded}

\includegraphics{GGplot-Iris_files/figure-latex/unnamed-chunk-1-6.pdf}

\begin{Shaded}
\begin{Highlighting}[]
\NormalTok{IrisPlot <-}\StringTok{ }\NormalTok{IrisPlot }\OperatorTok{+}\StringTok{ }\KeywordTok{coord_cartesian}\NormalTok{(}\DataTypeTok{xlim=}\KeywordTok{c}\NormalTok{(}\FloatTok{2.0}\NormalTok{, }\FloatTok{7.0}\NormalTok{), }\DataTypeTok{ylim=}\KeywordTok{c}\NormalTok{(}\FloatTok{2.5}\NormalTok{, }\FloatTok{3.5}\NormalTok{))  }\CommentTok{# zooms in   # set se=FALSE to turnoff confidence bands}

\NormalTok{IrisPlot <-}\StringTok{ }\NormalTok{IrisPlot }\OperatorTok{+}\StringTok{ }\KeywordTok{coord_cartesian}\NormalTok{(}\DataTypeTok{xlim=}\KeywordTok{c}\NormalTok{(}\FloatTok{2.0}\NormalTok{, }\FloatTok{7.0}\NormalTok{), }\DataTypeTok{ylim=}\KeywordTok{c}\NormalTok{(}\FloatTok{2.5}\NormalTok{, }\FloatTok{3.5}\NormalTok{))  }\CommentTok{# zooms in   # set se=FALSE to turnoff confidence bands}
\end{Highlighting}
\end{Shaded}

\begin{verbatim}
## Coordinate system already present. Adding new coordinate system, which will replace the existing one.
\end{verbatim}

\begin{Shaded}
\begin{Highlighting}[]
\KeywordTok{plot}\NormalTok{(IrisPlot)}
\end{Highlighting}
\end{Shaded}

\includegraphics{GGplot-Iris_files/figure-latex/unnamed-chunk-1-7.pdf}

\begin{Shaded}
\begin{Highlighting}[]
\NormalTok{################}
\CommentTok{#}
\CommentTok{# Add Title and Labels}
\NormalTok{IrisPlot <-}\StringTok{ }\NormalTok{IrisPlot }\OperatorTok{+}\StringTok{ }\KeywordTok{labs}\NormalTok{(}\DataTypeTok{title=}\StringTok{"PetalLength Vs PetalWidth"}\NormalTok{, }\DataTypeTok{subtitle=}\StringTok{"From Iris Dataset"}\NormalTok{, }\DataTypeTok{y=}\StringTok{"Petal Width of the flower"}\NormalTok{, }\DataTypeTok{x=}\StringTok{"Petal Length of the flower"}\NormalTok{, }\DataTypeTok{caption=}\StringTok{"PetalLengthWidhtRatioPoints"}\NormalTok{)}

\CommentTok{# or}

\NormalTok{IrisPlot <-}\StringTok{ }\NormalTok{IrisPlot }\OperatorTok{+}\StringTok{ }\KeywordTok{ggtitle}\NormalTok{(}\StringTok{"Petal Length Vs Petal Width"}\NormalTok{, }\DataTypeTok{subtitle=}\StringTok{"From Iris Dataset"}\NormalTok{) }\OperatorTok{+}\StringTok{ }\KeywordTok{xlab}\NormalTok{(}\StringTok{"Petal Length of the flower"}\NormalTok{) }\OperatorTok{+}\StringTok{ }\KeywordTok{ylab}\NormalTok{(}\StringTok{"Petal Width of the flower"}\NormalTok{)}
\KeywordTok{plot}\NormalTok{(IrisPlot)}
\end{Highlighting}
\end{Shaded}

\includegraphics{GGplot-Iris_files/figure-latex/unnamed-chunk-1-8.pdf}

\begin{Shaded}
\begin{Highlighting}[]
\NormalTok{################}
\CommentTok{#Adding color based on column}
\NormalTok{IrisPlot <-}\StringTok{ }\NormalTok{IrisPlot }\OperatorTok{+}\StringTok{  }\KeywordTok{geom_point}\NormalTok{(}\KeywordTok{aes}\NormalTok{(}\DataTypeTok{col=}\NormalTok{Species), }\DataTypeTok{size=}\DecValTok{3}\NormalTok{)}
\KeywordTok{plot}\NormalTok{(IrisPlot)}
\end{Highlighting}
\end{Shaded}

\includegraphics{GGplot-Iris_files/figure-latex/unnamed-chunk-1-9.pdf}

\begin{Shaded}
\begin{Highlighting}[]
\CommentTok{#setting up themes}
\NormalTok{IrisPlot <-}\StringTok{ }\NormalTok{IrisPlot }\OperatorTok{+}\StringTok{ }\KeywordTok{theme}\NormalTok{(}\DataTypeTok{legend.position=}\StringTok{"None"}\NormalTok{)  }\CommentTok{# remove legend}
\KeywordTok{plot}\NormalTok{(IrisPlot)}
\end{Highlighting}
\end{Shaded}

\includegraphics{GGplot-Iris_files/figure-latex/unnamed-chunk-1-10.pdf}

\begin{Shaded}
\begin{Highlighting}[]
\NormalTok{IrisPlot <-}\StringTok{ }\NormalTok{IrisPlot }\OperatorTok{+}\StringTok{ }\KeywordTok{theme}\NormalTok{(}\DataTypeTok{legend.position=}\StringTok{"bottom"}\NormalTok{)}
\KeywordTok{plot}\NormalTok{(IrisPlot)}
\end{Highlighting}
\end{Shaded}

\includegraphics{GGplot-Iris_files/figure-latex/unnamed-chunk-1-11.pdf}

\begin{Shaded}
\begin{Highlighting}[]
\NormalTok{IrisPlot <-}\StringTok{ }\NormalTok{IrisPlot }\OperatorTok{+}\StringTok{ }\KeywordTok{theme}\NormalTok{(}\DataTypeTok{legend.position=}\StringTok{"left"}\NormalTok{)}
\KeywordTok{plot}\NormalTok{(IrisPlot)}
\end{Highlighting}
\end{Shaded}

\includegraphics{GGplot-Iris_files/figure-latex/unnamed-chunk-1-12.pdf}

\begin{Shaded}
\begin{Highlighting}[]
\NormalTok{################}
\CommentTok{#changing color palette}
\NormalTok{IrisPlot <-}\StringTok{ }\NormalTok{IrisPlot }\OperatorTok{+}\StringTok{ }\KeywordTok{scale_colour_brewer}\NormalTok{(}\DataTypeTok{palette =} \StringTok{"Set1"}\NormalTok{)  }\CommentTok{# change color palette}
\KeywordTok{plot}\NormalTok{(IrisPlot)}
\end{Highlighting}
\end{Shaded}

\includegraphics{GGplot-Iris_files/figure-latex/unnamed-chunk-1-13.pdf}

\begin{Shaded}
\begin{Highlighting}[]
\NormalTok{################}
\CommentTok{#changing color x coordinate sequence}
\CommentTok{#IrisPlot  <- IrisPlot  + scale_x_continuous(breaks=seq(0, 0.1))}
\KeywordTok{plot}\NormalTok{(IrisPlot)}
\end{Highlighting}
\end{Shaded}

\includegraphics{GGplot-Iris_files/figure-latex/unnamed-chunk-1-14.pdf}

\begin{Shaded}
\begin{Highlighting}[]
\NormalTok{################}
\CommentTok{# Reverse axis for petal length}
\NormalTok{IrisPlot  <-}\StringTok{ }\NormalTok{IrisPlot  }\OperatorTok{+}\StringTok{ }\KeywordTok{scale_x_reverse}\NormalTok{()}
\KeywordTok{plot}\NormalTok{(IrisPlot)}
\end{Highlighting}
\end{Shaded}

\includegraphics{GGplot-Iris_files/figure-latex/unnamed-chunk-1-15.pdf}

\begin{Shaded}
\begin{Highlighting}[]
\NormalTok{################}
\CommentTok{#Setting up themes}
\NormalTok{IrisPlot <-}\StringTok{ }\NormalTok{IrisPlot  }\OperatorTok{+}\StringTok{ }\KeywordTok{theme_classic}\NormalTok{() }\OperatorTok{+}\StringTok{ }\KeywordTok{labs}\NormalTok{(}\DataTypeTok{subtitle=}\StringTok{"Classic Theme"}\NormalTok{)}
\KeywordTok{plot}\NormalTok{(IrisPlot)}
\end{Highlighting}
\end{Shaded}

\includegraphics{GGplot-Iris_files/figure-latex/unnamed-chunk-1-16.pdf}

\begin{Shaded}
\begin{Highlighting}[]
\NormalTok{IrisPlot <-}\StringTok{ }\NormalTok{IrisPlot  }\OperatorTok{+}\StringTok{ }\KeywordTok{theme_dark}\NormalTok{() }\OperatorTok{+}\StringTok{ }\KeywordTok{labs}\NormalTok{(}\DataTypeTok{subtitle=}\StringTok{"Classic Theme"}\NormalTok{)}
\KeywordTok{plot}\NormalTok{(IrisPlot)}
\end{Highlighting}
\end{Shaded}

\includegraphics{GGplot-Iris_files/figure-latex/unnamed-chunk-1-17.pdf}

\begin{Shaded}
\begin{Highlighting}[]
\NormalTok{IrisPlot <-}\StringTok{ }\NormalTok{IrisPlot  }\OperatorTok{+}\StringTok{ }\KeywordTok{theme_linedraw}\NormalTok{() }\OperatorTok{+}\StringTok{ }\KeywordTok{labs}\NormalTok{(}\DataTypeTok{subtitle=}\StringTok{"Classic Theme"}\NormalTok{)}
\KeywordTok{plot}\NormalTok{(IrisPlot)}
\end{Highlighting}
\end{Shaded}

\includegraphics{GGplot-Iris_files/figure-latex/unnamed-chunk-1-18.pdf}

\begin{Shaded}
\begin{Highlighting}[]
\NormalTok{################}
\CommentTok{# Adding a line chart}
\NormalTok{IrisPlot <-}\StringTok{ }\NormalTok{IrisPlot }\OperatorTok{+}\StringTok{  }\KeywordTok{geom_point}\NormalTok{(}\DataTypeTok{col=}\StringTok{"purple"}\NormalTok{,}\DataTypeTok{size=}\DecValTok{4}\NormalTok{) }\OperatorTok{+}\StringTok{ }\KeywordTok{geom_line}\NormalTok{()}
\KeywordTok{plot}\NormalTok{(IrisPlot)}
\end{Highlighting}
\end{Shaded}

\includegraphics{GGplot-Iris_files/figure-latex/unnamed-chunk-1-19.pdf}

\begin{Shaded}
\begin{Highlighting}[]
\CommentTok{#theme_linedraw}

\NormalTok{#################}
\NormalTok{IrisBoxPlot  <-}\StringTok{ }\KeywordTok{ggplot}\NormalTok{(iris, }\KeywordTok{aes}\NormalTok{(}\DataTypeTok{x=}\NormalTok{Sepal.Length, }\DataTypeTok{y=}\NormalTok{Sepal.Width))}
\KeywordTok{plot}\NormalTok{(IrisBoxPlot)}
\end{Highlighting}
\end{Shaded}

\includegraphics{GGplot-Iris_files/figure-latex/unnamed-chunk-1-20.pdf}

\begin{Shaded}
\begin{Highlighting}[]
\NormalTok{IrisBoxPlot  <-}\StringTok{ }\KeywordTok{ggplot}\NormalTok{(iris, }\KeywordTok{aes}\NormalTok{(}\DataTypeTok{x=}\NormalTok{Sepal.Length, }\DataTypeTok{y=}\NormalTok{Sepal.Width)) }\OperatorTok{+}\StringTok{ }\KeywordTok{geom_boxplot}\NormalTok{()}
\KeywordTok{plot}\NormalTok{(IrisBoxPlot)}
\end{Highlighting}
\end{Shaded}

\begin{verbatim}
## Warning: Continuous x aesthetic -- did you forget aes(group=...)?
\end{verbatim}

\includegraphics{GGplot-Iris_files/figure-latex/unnamed-chunk-1-21.pdf}

\begin{Shaded}
\begin{Highlighting}[]
\NormalTok{IrisBoxPlot  <-}\StringTok{ }\KeywordTok{ggplot}\NormalTok{(iris, }\KeywordTok{aes}\NormalTok{(}\DataTypeTok{x=}\NormalTok{Sepal.Length, }\DataTypeTok{y=}\NormalTok{Sepal.Width)) }\OperatorTok{+}\StringTok{ }\KeywordTok{geom_boxplot}\NormalTok{(}\KeywordTok{aes}\NormalTok{(}\DataTypeTok{colour =}\NormalTok{ Species))}
\KeywordTok{plot}\NormalTok{(IrisBoxPlot)}
\end{Highlighting}
\end{Shaded}

\includegraphics{GGplot-Iris_files/figure-latex/unnamed-chunk-1-22.pdf}

\begin{Shaded}
\begin{Highlighting}[]
\NormalTok{IrisBoxPlot  <-}\StringTok{ }\KeywordTok{ggplot}\NormalTok{(iris, }\KeywordTok{aes}\NormalTok{(}\DataTypeTok{x=}\NormalTok{Sepal.Length, }\DataTypeTok{y=}\NormalTok{Sepal.Width)) }\OperatorTok{+}\StringTok{ }\KeywordTok{geom_boxplot}\NormalTok{(}\KeywordTok{aes}\NormalTok{(}\DataTypeTok{colour =}\NormalTok{ Sepal.Length))}
\KeywordTok{plot}\NormalTok{(IrisBoxPlot)}
\end{Highlighting}
\end{Shaded}

\begin{verbatim}
## Warning: Continuous x aesthetic -- did you forget aes(group=...)?
\end{verbatim}

\includegraphics{GGplot-Iris_files/figure-latex/unnamed-chunk-1-23.pdf}

\begin{Shaded}
\begin{Highlighting}[]
\NormalTok{IrisBoxPlot  <-}\StringTok{ }\KeywordTok{ggplot}\NormalTok{(iris, }\KeywordTok{aes}\NormalTok{(}\DataTypeTok{x=}\NormalTok{Species, }\DataTypeTok{y=}\NormalTok{Sepal.Width)) }\OperatorTok{+}\StringTok{ }\KeywordTok{geom_boxplot}\NormalTok{(}\KeywordTok{aes}\NormalTok{(}\DataTypeTok{colour =}\NormalTok{ Sepal.Length))}
\KeywordTok{plot}\NormalTok{(IrisBoxPlot)}
\end{Highlighting}
\end{Shaded}

\includegraphics{GGplot-Iris_files/figure-latex/unnamed-chunk-1-24.pdf}

\begin{Shaded}
\begin{Highlighting}[]
\NormalTok{IrisPlot <-}\StringTok{ }\NormalTok{IrisPlot  }\OperatorTok{+}\StringTok{ }\KeywordTok{geom_boxplot}\NormalTok{(}\DataTypeTok{outlier.colour =} \StringTok{"red"}\NormalTok{, }\DataTypeTok{outlier.shape =} \DecValTok{1}\NormalTok{)}
\KeywordTok{plot}\NormalTok{(IrisPlot)}
\end{Highlighting}
\end{Shaded}

\begin{verbatim}
## Warning: Continuous x aesthetic -- did you forget aes(group=...)?
\end{verbatim}

\includegraphics{GGplot-Iris_files/figure-latex/unnamed-chunk-1-25.pdf}

\begin{Shaded}
\begin{Highlighting}[]
\NormalTok{####################}
\CommentTok{#Geom contour and geom tile}
\NormalTok{IrisContourPlot <-}\StringTok{ }\KeywordTok{ggplot}\NormalTok{(iris, }\KeywordTok{aes}\NormalTok{(Sepal.Length, Petal.Length)) }\OperatorTok{+}\StringTok{ }\KeywordTok{geom_density_2d}\NormalTok{()}
\KeywordTok{plot}\NormalTok{(IrisContourPlot)}
\end{Highlighting}
\end{Shaded}

\includegraphics{GGplot-Iris_files/figure-latex/unnamed-chunk-1-26.pdf}


\end{document}
